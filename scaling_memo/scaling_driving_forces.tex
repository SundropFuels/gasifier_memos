\documentclass[11pt,twocolumn]{article}
\usepackage{caption}
\usepackage{anysize}
\usepackage{fancyhdr}
\usepackage{graphicx}
\usepackage{subcaption}
\usepackage{color}
\usepackage{balance}
\usepackage{lipsum}
\usepackage{multirow}
\usepackage{multicol}
\usepackage{booktabs}

\marginsize{.75in}{.75in}{.75in}{1in}
\pagestyle{fancy}
\rhead{\today}
\lhead{\includegraphics[height=2.0cm]{logo.jpg}}
\rfoot{\thepage}
\cfoot{}
\renewcommand{\headrulewidth}{0pt} %removes line from fancy header
\renewcommand{\thispagestyle}[1]{} %placers header and footer on first page 
\renewcommand{\abstractname}{Summary}
\setlength{\columnsep}{25pt}
\date{}
\title{Laboratory Gasification Memo\\Empirical scale-up model for tubular biomass gasification \vspace{-6ex}}

\begin{document}

\twocolumn[
  \begin{@twocolumnfalse}
    \maketitle
    \begin{abstract}
    


    \end{abstract}
  \end{@twocolumnfalse}
]

\section*{Experimental Methods}



\section*{Results and Discussion}

\subsection*{Aggregate effects screening}
To identify the important factors influencing the effects of interest (i.e. carbon yield, methane yield, and tar loading), a screening analysis was performed. [COMPLETE]

\subsection*{Comparison of candidates for limiting phenomena}
From the high level exploration of the data from both the Longmont Laboratory and the RDF, it was clear that the process was driven by either minimum residence time or by maximum heat duty at the entrance to the gasifier.  A residence time dependence would imply kinetic limitations, either in transport of reactive species to the surface of the biomass particles or in reaction of those species at the surface.  A total heat duty dependence would imply that heat transfer to the reactive species was the key determinant in overall performance.  The actual underlying limitation can have important consequences for reactor design and scale-up, as discussed further below.  While the minimum residence time and the maximum heat duty were correlated through many of the experiments, there were enough runs with differences to draw inferences about the dominating physics.

\subsubsection*{Carbon yield}
Figure \ref{fig-tau_min_X_good_long} shows how carbon yield varied with the minimum residence time for two different temperatures in the Longmont Laboratory gasifier; Figure \ref{fig-tau_min_X_good_RDF} shows the analogous information for experiments performed at the RDF.  Color in the plots indicates the relative partial pressure of steam in the system, which will be discussed in more detail below.  While the carbon yield appeared to have a strong positive correlation with residence time at the Longmont Laboratory, there was strange behavior at residence times above [VALUE].  There, at both 1350 $^{\circ}$C and 1450 $^{\circ}$C, the carbon yield decreased in temperature with increasing residence time.  To check whether these points were somehow related in time or condition, potentially indicating a temporary departure of the system from steady conditions, key information about the runs is shown in Table \ref{tab-key_info_odd_points}.



\section*{Conclusion}



\newpage
\appendix
\onecolumn



\end{document}


\documentclass[11pt,twocolumn]{article}
\usepackage{caption}
\usepackage{anysize}
\usepackage{fancyhdr}
\usepackage{graphicx}
\usepackage{subcaption}
\usepackage{color}
\usepackage{balance}
\usepackage{lipsum}
\usepackage{multirow}
\usepackage{multicol}
\usepackage{booktabs}

\marginsize{.75in}{.75in}{.75in}{1in}
\pagestyle{fancy}
\rhead{\today}
\lhead{\includegraphics[height=2.0cm]{logo.jpg}}
\rfoot{\thepage}
\cfoot{}
\renewcommand{\headrulewidth}{0pt} %removes line from fancy header
\renewcommand{\thispagestyle}[1]{} %placers header and footer on first page 
\renewcommand{\abstractname}{Summary}
\setlength{\columnsep}{25pt}
\date{}
\title{Laboratory Gasification Memo\\Empirical scale-up model for tubular biomass gasification \vspace{-6ex}}

\begin{document}

\twocolumn[
  \begin{@twocolumnfalse}
    \maketitle
    \begin{abstract}
    


    \end{abstract}
  \end{@twocolumnfalse}
]

\section*{Introduction}

In any chemical reactor system, a variety of physical phenomena could act as the rate limiting step that determines the overall performance.  These could be determined by reaction kinetics (e.g. intrinsic surface kinetics, diffusion through ash layers or gas films) or, for endothermic systems, by heat transfer into the reacting materials.  Which phenomenon is dominant often changes as process conditions (e.g. temperature, pressure), reactant properties (e.g. concentrations, particle sizes), and transport conditions (e.g. turbulent vs. laminar, diffusion coefficients) change; this sometimes occurs all within the same unit.  From a design perspective, a clear understanding of the dominant phenomena at the desired scale and process conditions is critical to choosing the right direction when approaching the problem of scale-up. 

To improve performance in a system dominated by chemical kinetics, steps must be taken to increase both the rate of reaction and the residence time, including decreasing reactant velocities, decreasing particle size and improving porosity, and increasing reactant concentrations.  If heat transfer is the most important factor, then the only route to improved performance is in increasing the effectiveness of all contributing modes of heat transfer.  This would suggest pursuing increased radiation rates (i.e. higher emissivities, near-critical optical thicknesses) and improved convection (i.e. higher velocities, better thermal mixing through turbulence).  These different approaches can lead to drastically different designs in a scaled-up plant.  Most importantly, a designer seeking to maximize reaction kinetics would minimize her surface area to volume ratio, while one attemping to maximize heat transfer would try to achieve the opposite.

In order to determine the controlling phenomenon in Sundrop Fuels' RPReactor\texttrademark gasifier technology, data from both the Reactor Development Facility (RDF) pilot plant and Longmont Laboratory small gasifier were surveyed.  As these data come from a variety of experimental campaigns, each with differing objectives, they do not form as controlled an experiment as one would wish for explicitly testing the controlling effects.  However, the data is voluminous, and they were taken over a wide range of conditions at two widely differing scales.  The goal of this effort was, with appropriate filtering and analysis, to determine a clear picture of which phenomena should guide design as well as to recommend any further investigation that will be required to clarify and confirm (or reject) its findings.

\section*{Experimental Methods}



\section*{Results and Discussion}

\subsection*{Overall data survey}
To get a general feel for the ranges and trends in the data at both scales, some survey data is presented here.  First, the variation in the data over time was examined to determine whether there were system changes during that would confound interpretation of the other system variables.  Figure \ref{fig-Lab-Xg-dates} shows the carbon yield for the laboratory gasifier over the 18 months of operation of that system.  While the variation was greater in the first 6 months of operation than in the past year, there are no clear trends or structure to the data over the course of operation.  It will be seen below that the variation in the first 6 months can be attributed to other factors.  Likewise, an examination of the methane yield (Figure \ref{fig-Lab-CH4-dates}) shows no significant structure or change in variation during the period of operation.  Figures \ref{fig-RDF-Xg-dates} and \ref{fig-RDF-CH4-dates} show the analogous information for the RDF.  It is important here to recall that the RDF campaigns were run at a range of temperatures across its 8 months of main operation, starting with SiC between 1400 $^{\circ}$C and 1450 $^{\circ}$C and concluding with a metal tube at 1200 $^{\circ}$C.  These temperature ranges are color-coded in the figures.  Within the temperature series for carbon yield, there is no structure at 1200 ${^\circ}$C but some reduction in variation at the higher temperatures.  As will be seen, this corresponded with the introduction of the de-gasser; there were few replicates to verify consistent performance across time. [CH4 data]

[REPLICATION ANALYSIS]

Figure \ref{fig-Lab-Xg-mdot} shows the variation in carbon yield with mass flow of the biomass for the Longmont Laboratory gasifier.  Mass flow is generally a good proxy for overall heat duty.  However, in this system, entraining gas requirements are strongly dependent on the mass flow rate, and the gasification reaction itself generates a significant amount of gas.  Both of these would suggest that $t_{min}$ also decreases with increasing mass flow rate, confounding the two effects.  The trends in the data were as one would suspect for either controlling factor.  Increasing mass flow rate led to a decrease in the overall carbon yield, and higher temperatures and smaller particles led to higher carbon yields.  There were relatively low levels of variance in similar temperature/particle size points at each mass flow rate (5\%-10\%) between 1.5 lb/hr and 4.0 lb/hr, suggesting some other reaction conditions had minor effects on performance.  However, at 1.0 lb/hr there was a wide range (0.5-0.95) of carbon yield, indicating that there were a number of other factors at work in those experiments.  As the 1.0 lb/hr experiments had the greatest potential for choosing process conditions (i.e. the minimum required entrainment gas flow rates were the lowest), it was likely that there was further differentiation between kinetic and heat transfer indicating variables there.  This was explored in detail in the next section.  Figure \ref{fig-Lab-CH4-mdot} shows the methane yield as a function of mass flow rate of the biomass.  It was, as well, typical of what one would expect for either set of controlling physics.  Of note is that there was significant separation between the temperature curves, much more than seen for the carbon yield.  This was likely due to the high temperature kinetics of the methane decomposition and hydro-methanation reactions and the changes in water-gas shift equilibrium at those temperatures.  This is discussed in some detail below, but will be addressed in much more detail in a later memo.  Also of note is that the spread in the data is again much greater for the 1.0 lb/hr case, suggesting that there were more interesting effects than just indicated by the mass flow rate.  Again, this was be a focus of the investigation below.  Finally, it should be noted that the current economic simulations assume a methane mole fraction in the syngas of 0.5\%, which translates to just about a 2\% methane yield.  Only conditions at a nominal wall temperature of 1450 $^{\circ}$C achieved this requirement.  If lower operating temperatures are desired, something must be done to fundamentally change reactor performance.

Figure \ref{fig-RDF-Xg-mdot} shows the trends in carbon yield for the RDF data with mass flow rate of biomass.  The trends are much as one would expect.  The data is structured in downward trending series, with the series designated by the temperature of the tube wall, and the conversions range from around 85\% at the lowest biomass flowrates/highest temperatures to 40\% for the lowest temperatures.  As would be expected for either heat transfer or kinetic controls, the higher temperature series have generally higher conversion than the lower temperature series.  Unfortunately, mass flow rate is not a definitive proxy for either controlling mode.  Increasing mass flow rate would be expected to decrease residence time, as additional entraining gas and reaction products would increase velocities throughout the tube.  Increasing mass flow rate would clearly increase overall reactor heat duty as well.  To distinguish the effects, a closer examination was undertaken in the next section.  Figure {fig-RDF-CH4-mdot} shows the variation in the methane yield as a function of mass flow rate of biomas for the RDF data.  In stark contrast to the laboratory data, there is no clear trend in the methane yield with mass flowrate of biomass for any of the temperatures displayed.  It remains unclear why this was, when expectations would suggest increasing conversion of CH4 with increasing levels of carbon yield; the data is explored further to this end later in this memo.  

In addition to overall mass flowrate of biomass, space time is a typical measure for general reactor performance.  For a plug flow reactor with no volume expansion, it is directly related to the integrated rate; kinetically controlled systems generally have performance curves that increase at ever decreasing rates with space time.  When volume expansion is included, the relationship is more complicated, but the same trends seem to apply.  Figure \ref{fig-Lab-Xg-tau} shows the carbon yield for the Longmont Laboratory gasifier as a function of space time; temperature is indicated by color and particle size is indicated by marker size.  The trends were clearly as one would expect for a gasification reaction.  There was an increase in carbon yield with space time, and that increase slowed as carbon yield approached 90\%.  Moreover, there were series contained within the data: increased temperature lead to higher carbon yields at a given space time, and the trends for the largest particles were at the lowest end of each of these series.  Looking closer, there was a significant level of spread in the data: for the 1450 $^{\circ}$C data, there is a range of 10\%--15\%, much greater than the repeatability level of [REPEAT FIG].  This indicated that if space time was a major factor, there were other important factors causing the additional variance within each of the temperature series.  This additional variance was a major focus of the analysis described below.  Additionally, the carbon yield appeared to saturate between 3.0 s and 4.0 s of space time, and it saturated at different levels for each temperature.  If not a spurious effect, this would indicate either a shift in reaction kinetics (with strong temperature dependence) or an equilibrium limitation.  If kinetic control were indicated by further analysis, this would suggest that detailed investigation of this late stage would be critical for pushing carbon yield to maximal levels. 

Figure \ref{fig-Lab-CH4-tau} shows the dependence of methane yield on space time.  The trends were, as with the carbon yield, what one might expect for kinetic control of performance.  In general, increases in space time led to reductions in the methane yield, with higher temperatures and smaller particle sizes giving lower levels as well.  However, the trends are not completely monotonic.  At low space time (~0.5 s), the 1350 $^{\circ}$C series shows lower values in yield than its counterparts at 1.5 s - 2.0 s.  Methane is generated early in the set of pyrolysis and gasification reactions, so a maximum between these two points (missing in the data) could indicate this behavior.  The 1450 $^{\circ}$C series may have shown a maximum near 2.0 s as well.  However, it would be expected that this series would achieve its maximum at \emph{lower} space time than the 1350 $^{\circ}$C series, on the account of the expected exponential increase in kinetics with temperature.  Additionally, there are a set of points at 0.5 s with much higher levels of methane yield.  An alternate explanation would be that space time was a poor proxy for residence time, and that $t_{min}$ would show more clear structure in the data.  Finally, it is possible that the deviations in the expected structure of the data are signs that the system was not really under kinetic control.

The space time effect is examined for the RDF in Figures \ref{fig-RDF-Xg-tau} and \ref{fig-RDF-CH4-tau}.  Clearly, as shown before, the higher temperature experiments resulted in higher carbon yields.  However, at both the higher and the lower temperatures, there appeared to be no effect at all of the space time on the carbon yield.  As space time is, in general, a good proxy for the residence time, this is a first suggestion that the residence time (and thus, the reaction kinetics) may not be the controlling factor.  An argument could be made that gas release from the biomass is not a constant function of reaction progress; following that line of reasoning, a large fraction of the product gas could be released relatively early in the reactor, skewing some high biomass flowrate results with long spacetimes to shorter residence times.  For kinetic controls, it is the residence time that truly matters, so the space time metric may obfuscate a real residence time effect.

To work around this objection, a ``minimum residence time" ($t_{min}$) was calculated, as was described above.  Because this was based on the outlet gas flowrate as measured in the system, it provides a lower bound on the residence time in the system.  Space time provides the corresponding upper bound.  Because models of the pyrolysis phase of the reaction [REF] have shown a large fraction of the total gas release at low temperatures [RANGE], it is likely that $t_{min}$ provided a better estimate of the actual residence time in the system than $\tau$.  In any case, if neither showed a strong correlation with the gasifier performance, it would be unlikely that reaction kinetics were the dominant phenomenon.

\subsection*{Comparison of candidates for limiting phenomena}

As the general overview of the data from the Longmont Laboratory and the RDF made clear, there was significant overall variation in the data and clear trends with some superficial variables.  Further analysis was made to move beyond these superficial variables to those that most closely cohere with the controlling phenomena and that would allow clearest distinction between them.  For kinetic controls, this variable was the minimum residence time ($t_{min}$), as described above.  For heat transfer controls, this was the surface specific heat duty of the reactor ($\frac{\Delta H_{max}}{A}$), also described above.  

In the Longmont Laboratory system, these variables were themselves closely correlated. It is clear from Figure \ref{fig-Lab-dH_max-tmin} that most of the points lie within a tight band that suggested a monotonically decreasing connection between the variables, and the correlation coefficient was [CORR COEFF].  This was a result of two factors: there was a minimum entrainment gas flow rate that was a strong function of the biomass flow rate, and the experiments as designed were looking at a range of factors independently (e.g. space time, mass flow rate, pressure, temperature, particle size) but were not explicitly looking at the heat duty.  It is a recommendation of this memo that controlled experiments exploring these variables be performed to confirm the conclusions of this document.  A key result of this was that, for many of the data views explored below, it could be reasonably inferred that either kinetic control or heat transfer control was important.  However, given the limitations of the current data set, there were three areas on the correlation plot (denoted Slice 1, Slice 2, and Slice 3) where there was a large change in one variable with only small changes in the other variable.  These were exploited in the analysis below to gain some insight into the possible controlling factors.  This creative technique was applied to other data plots as the individual effects were examined in more detail.

At the RDF, the separation between the variables was much more distinct (Figure \ref{fig-RDF-dH_max-tmin}).  At 1450 $^{\circ}$C there was essentially no correlation between the variables, providing an excellent opportunity for testing the effects independently.  Likewise, at 1250 $^{\circ}$C there is only minor correlation, and the correlation is positive.  This is in the opposite direction of the Longmont Laboratory data and again provides a good opportunity for independent comparison.  At 1400 $^{\circ}$C the variables do appear to be correlated somewhat [CORR COEFF] in a similar fashion to the Longmont Laboratory data, providing less opportunity for discrimination.

\subsubsection*{Investigation of kinetic controls}

$t_{min}$ was used as the chief proxy of actual residence time.  For the Longmont Laboratory gasifier, Figure \ref{fig-Lab-Xg-tau-temps} shows the effect of $t_{min}$ on carbon yield for the two temperatures with the highest density of data (1350 $^{\circ}$C and 1450 $^{\circ}$C).  From a first look, it appeared that the system performed exactly as one would have expected for a kinetically controlled system.  Higher temperatures gave higher carbon yields at similar $t_{min}$ values, with the difference being around 10\%.  Increasing residence time gave increasing conversion in a saturating curve, and the larger particles followed a distinct series from the smaller particles.  The saturation was at ~90\% for the 1450 $^{\circ}$C experiments and between 80\%--85\% for the 1350 $^{\circ}$C experiments, again suggesting a kinetic shift or an equilibirium limitation.

A closer look, however, raised doubts about these conclusions.  First, the spread in the data was relatively large, especially in the 1450 $^{\circ}$C plot: 10--15 percentage points at $t_{min}$ between 0.1 s and 0.5 s, and nearly 30 percentage points at the highest residence times.  This variance at a minimum suggested secondary effects, and it hinted at the possibility that minimum residence time may not be the right controlling variable.  As shown on the plot, data sets with very close $t_{min}$ values but large spreads in the carbon yield were selected.  For the set labelled ``Slice 1", Figure \ref{fig-Lab-Xg-tmin-temps-slice1} shows the variation of carbon yield with the heat transfer proxy, $\frac{\Delta H_{max}}{A}$.  There is a 50\% variation in this variable, and the carbon yield showed a clear linear relationship to it.  [OTHER SLICES] [possible secondary effects/other explanations]

Further examination showed that there was a curious dip in the carbon yield at $t_{min}$ between 0.5 s and 0.6 s for both temperature curves, which then increased again at the saturation level between 0.6 s and 0.7 s.  While this could have been noise in the data, repeatability metrics for the Longmont Laboratory have been exceptionally good [MEASURE], suggesting that effects seen were real.  Unless there would exist reactions that would reduce the levels of both CO and CO$_{2}$ that initiate at higher temperatures, there would be no way for a kinetically controlled system to show this temporary decrease in carbon yield with increasing residence time.  As these are the oxides of carbon with the lowest enthalpy levels and highest entropy levels, it would be surprising if such reactions existed.  The saturation behavior itself was also somewhat surprising; at 1350 $^{\circ}$C, an increase of 40\% in the minimum residence time led to essentially no change in carbon yield; a similar effect was seen at 1450 $^{\circ}$C.  There was a large difference (90\% vs 80\%) in the final conversion levels.  Not only would this suggest a substantial and sudden decrease in the reaction rate, the difference in saturation levels for temperature would suggest that the new controlling kinetics had an extraordinately large activation energy giving quasi-equlibrium levels of carbon yield.  None of these effects on their own would be convincing evidence against a kinetic explanation for the data, but taken together they suggest that an alternate explanation be investigated.  This is especially bolstered by the sliced data, which showed a strong dependence on the proxy variable for heat transfer control, a leading candidate for a different explanation of the data.

As was mentioned above, the RDF data provided a better opportunity to distinguish between the two candidate controlling phenomena, as $t_{min}$ and $\frac{\Delta H_{max}}{A}$ were not tightly correlated for any of the temperature series.  Figure \ref{fig-RDF-Xg-tmin} shows the effect of minimum residence time on carbon yield, split into different plots for each major temperature series (1450 $^{\circ}$C, 1400 $^{\circ}$C, and 1200 $^{\circ}$C.  It became clear in these plots that any potential correlation between the carbon yield and the minimum residence time was unlikely; all three series had what appeared to be randomly distributed data.  There might be a slight upward trend for the 1400 $^{\circ}$C series, but there are two points (with 0.85 and 0.81 carbon yields) that have very high carbon yields for relatively short residence times.  The other two series, 1200 $^{\circ}C$ and 1450 $^{\circ}$C show no trend at all.  Although the data from the Longmont Laboratory made a less compelling argument for the lack of kinetic control, the RDF data made a fairly strong case that residence time was not a determining factor.

Another key metric for performance is the yield of methane, which should be below 2\% to match the current commercial scale design for economic reasons.  Figure \ref{fig-Lab-CH4-tmin-temps} shows the methane yield for both 1450 $^{\circ}$C and 1350 $^{\circ}$C operating conditions as a function of minimum residence time.  As with the carbon yield, the trends were much where they would be expected for the Lab data for kinetic control on a first examination.  Higher temperatures yielded less methane, and larger particles perform somewhat worse than small particles.  From a kinetic perspective, it was not entirely clear why this would be the case, as the methane decomposition reactions should not be surface specific and hydro-methanation reactions should be; if anything, lower levels of methane might be expected for larger particles.  There was a significant spread in the data at similar residence times, encompassing up to 2.5 percentage points at some values of $t_{min}$.  Again, due to the repeatability of the data [VALUE], this suggested either a secondary effect at work or a spurious correllation.  Closer examination revealed further pathology.  In the 1450 $^{\circ}$C data, there is a ``cliff" at 0.5 s of minimum residence time where the methane yield suddenly goes from between 0.00 and 0.02 to only values of 0.00.  This lack of variation suggested that either the points beyond 0.5 s had a constant value of the variable driving the secondary effect, or that the overall association with residence time was false. [LOOK AT SECONDARY VARIABLES]  The data at 1350 $^{\circ}$C showed the expected downard trend in methane yield, but also had an interesting \emph{increase} in methane yield between 0.1 s and 0.3 s.  Likewise, there were a number of points well below the general trend at 0.3 s.  These deviations could be a sign of a maximum in methane yield between 0.1 s and 0.3 s -- they should certainly be compared with the kinetic models, and some experimental points run in the space with few points.  They could also signal that the strong trend seen in all of the other data points was instead the result of the overall strong coupling between the surface specific heat duty and the minimum residence time for most of these experiments.  The latter would align well with the conclusions drawn from the carbon yield data at both the Longmont Laboratory and the RDF.

Methane yield at the RDF was, in general, significantly more difficult to interpret than in the Longmont Laboratory.  For starters, it was much higher (9\% - 18\%) than in the Longmont Laboratory.  It also did not show the expected trends.  Figure \ref{fig-RDF_CH4-tmin-temps} shows the data.  At 1200 $^{\circ}$, it \emph{increases} with minimum space time; this was entirely surprising, as increasing time in the ranges (0.3 s -- 1.1 s) should allow the methane to be further destroyed rather than accumulated.  At 1450 $^{\circ}$C, there did not appear to be a trend at all in the data.  It was clear that, at least for methane decomposition, this was not an effect controlled by residence time at the RDF.

[AGGREGATE SLICES FROM TAU/DH CURVE]

Finally, to compare both the Longmont Laboratory and the RDF data, an aggregated data curve is shown for carbon yield at the only common operating temperature between the two scales (1450 $^{\circ}$C, Figure \ref{fig-Agg-Xg-tmin-1450}).  It can be clearly seen that there was no scale correspondence between the Longmont Laboratory and the RDF; that is, if kinetic controls played a key role, the nature of those controls would have changed with the scale, greatly diminishing the value of models developed at a laboratory level for predicting the performance at larger levels.  However, there is a more likely explanation.  Given the preponderance of data presented in this section, it could most reasonably be concluded that residence time (and thus, a kinetic phenomenon) was not the significant factor for determining performance at the range of scales and process conditions examined in these experiments.  Further experimentation will be done to confirm this conclusion, but the evidence is very strong that kinetics were not the set of performance determining physics.

\subsubsection*{Investigation of heat transfer controls}

If kinetic concerns were not the driving factors behind reactor performance, perhaps the system was controlled by heat transfer.  As explained above, the maximum achievable enthalpy change per surface area ($\frac{\Delta H_{max}}{A}$ was used as a surface specific heat duty; this is similar to the duty used in sizing a heat exchanger.  If the specific enthalpy of the system due to sensible heat changes and heats of reaction is a linear function of temperature, as has been seen in some of the thermodynamic simulations [REF], it would be expected that performance would decline linearly with increasing heat loading.  These effects were examined for both scales in the analysis that follows.



[Lab for dHmax]
[Go into secondary effects here]

[RDF for dHmax]

[Lab for methane yield]

[RDF for methane yield]

[Lab for tar yield]

[Collapse for T4]

[Aggregated lab/RDF data -- coup d' grace]


\section*{Conclusion}



\newpage
\appendix
\onecolumn



\end{document}

